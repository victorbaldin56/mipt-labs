\documentclass[12pt]{article}
\usepackage[utf8]{inputenc}
\usepackage[english, russian]{babel}
\usepackage{graphicx, float, multicol, hyperref}

\title{Статистическая обработка результатов многократных измерений}
\author{Балдин Виктор}

\begin{document}
    \maketitle
    \section{Аннотация}
    \par \textbf{Цель работы:} применение методов обработки
    экспериментальных данных при измерении сопротивлений.
    \par \textbf{В работе используются:} набор резисторов (270 штук); 
    универсальный цифровой вольтметр GDM-8145, работающий в режиме 
    <<Измерение сопротивление постоянному току>>.
    \section{Теоретические сведения}
    \par Производство резисторов на заводе -- сложный технологический процесс. 
    Поэтому измеренное сопротивление может отличаться от номинала. Погрешности 
    могут быть как систематическими, так и случайными.
    \par Для измерения сопротивления мы будем пользоваться прибором, погрешность
     которого мала  $\left(  \pm 0,5 \mbox{ Ом} \right) $ по сравнению с 
     отклонениями от номинала, полученными при производстве. Поэтому cистематической 
     погрешностью можно пренебречь.

    \par В работе измеряем сопротивление 270 резисторов. 
    По полученным данным вычисляем среднее значение:

    \begin{equation}\label{for1}
        \langle R \rangle = \frac{1}{N} \sum_{i=1}^N R_i.
    \end{equation}\\

    \par Чтобы охарактеризовать случайные погрешности при изготовлении набора резисторов, 
    необходимо построить гистограмму. Для этого разделим интервал значений сопротивления на 
    $m$ равных частей:
    
    \begin{equation}
        \Delta R = \frac{R_{\mbox{\tiny{max}}} - R_{\mbox{\tiny{min}}}}{m}
    \end{equation}

    \par По оси $y$ гистограммы отложим плотность вероятности
    \begin{equation}
        y = \frac{\Delta n}{N\Delta R},
    \end{equation}
    где $\Delta n$ -- число измерений, попадающих в заданный интервал.
    \par Среднеквадратичное отклонение можно найти как:
    \begin{equation}
        \sigma = \sqrt{\frac{1}{N}\sum_{i = 1}^{n}(R_i - \langle R\rangle)^2}
    \end{equation}

    \par Построим функцию распределения Гаусса:
    \begin{equation}
        y = \frac{1}{\sqrt{2\pi}\sigma}e^{-\frac{(R - \langle R\rangle)^2}{2\sigma^2}}
    \end{equation}

    \section{Методика измерений}
    Измерения будем проводить при помощи универсального мультиметра GDM-8145, погрешностью
    прибора при этом пренебрежем в силу ее малости по справнению со случайным разбросом.

    \section{Результаты измерений и обработка данных}
    
\end{document}