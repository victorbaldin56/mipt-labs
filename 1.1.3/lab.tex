\documentclass[12pt]{article}
\usepackage[utf8]{inputenc}
\usepackage[english, russian]{babel}
\usepackage{graphicx, float, multicol, hyperref, pgfplots}

\title{Статистическая обработка результатов многократных измерений}
\author{Балдин Виктор}

\begin{document}
    \maketitle
    \section{Аннотация}
    \par \textbf{Цель работы:} применение методов обработки
    экспериментальных данных при измерении сопротивлений.
    \par \textbf{В работе используются:} набор резисторов (270 штук);
    универсальный цифровой вольтметр GDM-8145, работающий в режиме
    <<Измерение сопротивление постоянному току>>.
    \section{Теоретические сведения}
    \par Производство резисторов на заводе -- сложный технологический процесс.
    Поэтому измеренное сопротивление может отличаться от номинала. Погрешности
    могут быть как систематическими, так и случайными.
    \par Для измерения сопротивления мы будем пользоваться прибором, погрешность
     которого мала  $\left(  \pm 0,5 \mbox{ Ом} \right) $ по сравнению с
     отклонениями от номинала, полученными при производстве. Поэтому cистематической
     погрешностью можно пренебречь.

    \par В работе измеряем сопротивление 270 резисторов.
    По полученным данным вычисляем среднее значение:

    \begin{equation}\label{for1}
        \langle R \rangle = \frac{1}{N} \sum_{i=1}^N R_i.
    \end{equation}\\

    \par Чтобы охарактеризовать случайные погрешности при изготовлении набора резисторов,
    необходимо построить гистограмму. Для этого разделим интервал значений сопротивления на
    $m$ равных частей:

    \begin{equation}
        \Delta R = \frac{R_{\mbox{\tiny{max}}} - R_{\mbox{\tiny{min}}}}{m}
    \end{equation}

    \par По оси $y$ гистограммы отложим плотность вероятности
    \begin{equation}
        y = \frac{\Delta n}{N\Delta R},
    \end{equation}
    где $\Delta n$ -- число измерений, попадающих в заданный интервал.
    \par Среднеквадратичное отклонение можно найти как:
    \begin{equation}
        \sigma = \sqrt{\frac{1}{N}\sum_{i = 1}^{n}(R_i - \langle R\rangle)^2}
    \end{equation}

    \par Построим функцию распределения Гаусса:
    \begin{equation}
        y = \frac{1}{\sqrt{2\pi}\sigma}e^{-\frac{(R - \langle R\rangle)^2}{2\sigma^2}}
    \end{equation}

    \section{Методика измерений}
    Измерения будем проводить при помощи универсального мультиметра GDM-8145, погрешностью
    прибора при этом пренебрежем в силу ее малости по справнению со случайным разбросом.

    \section{Используемое оборудование}
    Набор резисторов (270 штук); мультиметр GDM-8145, работающий в режиме
    измерения сопротивления постоянному току.

    \section{Результаты измерений и обработка данных}
    Результаты измерения сопротивлений (в Омах) удобно представить в таблице:\\

    \begin{table}[]
        \caption{Результаты измерения сопротивления 270 резисторов (в Ом)}
        \begin{tabular}{|c|c|c|c|c|c|c|c|c|c|}
        \hline
        499.9 & 499.9 & 499.8 & 500.1 & 499.9 & 499.6 & 499.6 & 490.8 & 499.8 & 499.1 \\ \hline
        500.7 & 499.9 & 501.4 & 498.5 & 498.6 & 499.8 & 498.9 & 498.6 & 499.8 & 502.0 \\ \hline
        497.8 & 501.6 & 497.8 & 503.7 & 498.9 & 499.0 & 499.0 & 500.2 & 499.4 & 499.4 \\ \hline
        498.8 & 501.9 & 503.0 & 499.6 & 501.0 & 498.7 & 108.3 & 499.1 & 499.3 & 498.8 \\ \hline
        503.3 & 498.9 & 500.0 & 501.3 & 499.3 & 497.5 & 498.2 & 499.7 & 497.9 & 501.0 \\ \hline
        501.3 & 499.7 & 499.8 & 497.6 & 498.8 & 499.5 & 501.1 & 498.7 & 500.0 & 498.3 \\ \hline
        500.2 & 499.8 & 498.3 & 498.0 & 498.3 & 499.9 & 499.6 & 500.6 & 499.2 & 497.7 \\ \hline
        498.2 & 498.3 & 499.3 & 501.7 & 502.3 & 500.2 & 500.7 & 501.0 & 500.7 & 497.5 \\ \hline
        501.2 & 499.6 & 500.1 & 500.0 & 499.4 & 500.1 & 501.8 & 501.8 & 497.0 & 500.8 \\ \hline
        499.2 & 500.0 & 499.6 & 499.5 & 500.0 & 498.0 & 500.0 & 501.1 & 498.2 & 498.0 \\ \hline
        500.9 & 497.5 & 497.6 & 497.6 & 500.2 & 499.5 & 500.1 & 500.9 & 499.4 & 496.6 \\ \hline
        498.7 & 500.0 & 501.1 & 499.8 & 499.0 & 499.2 & 499.4 & 499.0 & 500.2 & 501.5 \\ \hline
        498.6 & 498.9 & 501.7 & 501.9 & 500.0 & 498.9 & 499.9 & 498.7 & 498.3 & 500.9 \\ \hline
        498.4 & 498.2 & 499.8 & 500.4 & 497.8 & 499.8 & 499.6 & 498.9 & 501.5 & 497.9 \\ \hline
        497.9 & 499.3 & 500.9 & 498.7 & 499.2 & 499.6 & 502.0 & 499.0 & 499.0 & 500.2 \\ \hline
        501.6 & 500.2 & 501.6 & 498.2 & 497.8 & 499.2 & 498.7 & 499.9 & 499.3 & 499.0 \\ \hline
        499.2 & 498.6 & 500.0 & 497.3 & 499.5 & 498.3 & 499.8 & 499.5 & 500.6 & 500.0 \\ \hline
        499.3 & 500.5 & 498.6 & 497.8 & 498.5 & 500.9 & 498.9 & 501.6 & 500.5 & 500.5 \\ \hline
        499.9 & 499.7 & 497.0 & 502.0 & 501.6 & 501.5 & 500.2 & 500.6 & 499.5 & 501.5 \\ \hline
        500.3 & 499.8 & 501.4 & 501.6 & 500.5 & 499.1 & 498.5 & 500.5 & 499.2 & 500.2 \\ \hline
        499.1 & 499.0 & 497.6 & 499.6 & 501.0 & 500.5 & 500.6 & 499.8 & 500.3 & 498.2 \\ \hline
        501.4 & 498.9 & 500.1 & 501.5 & 499.3 & 497.5 & 499.3 & 499.6 & 499.6 & 499.0 \\ \hline
        501.9 & 501.4 & 501.4 & 500.0 & 499.3 & 501.6 & 499.6 & 499.8 & 496.8 & 498.2 \\ \hline
        499.0 & 499.3 & 500.0 & 499.3 & 506.2 & 498.8 & 498.4 & 499.6 & 503.2 & 499.8 \\ \hline
        499.7 & 502.1 & 498.8 & 499.2 & 499.0 & 499.4 & 498.0 & 497.9 & 497.8 & 497.5 \\ \hline
        502.6 & 501.7 & 500.1 & 500.0 & 500.7 & 500.3 & 499.6 & 501.7 & 500.6 & 500.6 \\ \hline
        502.4 & 499.3 & 500.9 & 499.2 & 501.0 & 499.1 & 499.2 & 497.8 & 498.7 & 498.0 \\ \hline
        499.1 & 499.2 & 500.5 & 499.2 & 500.4 & 498.8 & 498.9 & 499.4 & 498.0 & 499.1 \\ \hline
        \end{tabular}
    \end{table}

    Построим гистограмму для $m = 20$:
    \begin{figure}[H]
        \centering
        \input{graphs/hist.pgf}
        \caption{Гистограмма плотности вероятности для $m = 20$}
    \end{figure}

    \par Вертикальные прямые на гистограммах показывают границы ключевых
    для распределения Гаусса (или нормального распределения) интервалов
    $\langle R \rangle \pm \sigma$, $\langle R \rangle \pm 2\sigma$ и
    $\langle R \rangle \pm 3\sigma$.
    \par Теперь возьмем половину от количества измеренных значений и построим
    гистограмму для них:
    \begin{figure}[H]
        \centering
        \input{graphs/my_hist.pgf}
        \caption{Гистограмма для меньшего числа значений}
    \end{figure}

    Для наглядности анализа получившихся
    результатов представим доли значений из
    разных гистограмм, попавших в <<ключевые>> интервалы,
    таблицей:
    \begin{table}[H]
    \caption{Доли значений, лежащих в интервалах, для разных гистограмм}
    \begin{tabular}{|c|c|c|c|}
        \hline
        Интервал                       & I  & II \\ \hline
        $\langle R\rangle \pm \sigma$   & 67\%  & 65\% \\ \hline
        $\langle R \rangle \pm 2\sigma$ & 97\%  & 97\% \\ \hline
        $\langle R \rangle \pm 3\sigma$ & 100\% & 99\% \\ \hline
        \end{tabular}
    \end{table}

    \section{Обсуждение результатов}
    Как видно из гистограмм, плотность вероятности приблизительно удовлетворяет
    теоретической зависимости $y(R)$. При этом увеличение числа измерений
    позволило добиться лучшего соответствия.

    \section{Выводы}
    Из результатов наших измерений видно, что погрешность в сопротивлении
    резисторов вызвана преимущественно случайным разбросом, т. к. можно
    наблюдать хорошее соответствие нормальному распределению. Величина этого
    разброса позволяет заключить, что точное производство резисторов -- довольно
    сложная технологическая задача.

\end{document}
